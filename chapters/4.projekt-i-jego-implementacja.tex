\chapter{Projekt i jego implementacja}

Założeniami projektu opisanego w ninejszej pracy było stworzenie prototypu
oprogramowania umożliwiającego wykrywanie ataków DDoS w sieciach SDN oraz, co
ważniejsze, dającego się łatwo skalować w celu zwiększenia jego wydajności.
Wykorzystanie właściwej archtiektury rozwiązania oraz dobór odpowiednich
technologii do jego implementacji umożliwiły osiągnięcie zamierzonych celów.

\section{Architektura rozwiązania ochrony przed atakiem DDoS}

Projektując tytułowe rozwiązanie należało wziąć pod uwagę następujące czynniki:
\begin{itemize}
  \item umiejscowienie komponentu z oprogramowaniem w architekurze sieci SDN
  \item uzyskanie danych z sieci, pozwalających na wykrycie ataku
  \item wydajność działania oprogramowania
\end{itemize}

Oprogramowanie zostało zaprojektowane do działania na południowym interfejsie
architektury sieci SDN. Takie umiejscowienie oprogramowania dostarcza wielu
możliwości. Pierwszą z nich jest to, że oprogramowanie może być częścią
przełącznika, kontrolera lub działać na osobnym węźle. Dodatkowo oprogramowanie
ma dostęp do danych wymienianych pomiędzy przełącznikiem, a kontrolerem.
Możliwość analizy tychże danych była podstawą do implementacji algorytmu do
wykrywania ataków DDoS w sieci. Szczegóły dotycznące algorytmu zostały opisane w
późniejszym rozdziale ninejszej pracy.

Wspomniane rozwiązanie zaimplementowano z myślą o działaniu na osobnym węźle,
ponieważ dzięki takiemu podejściu, w przypadku równoległego działania
oprogramowania, liczba jego instancji jest niezależna od liczby przełączników
lub kontrolerów. Dzięki temu, możliwe jest bardziej elastyczne skalowanie
horyzontalne oprogramowania w celu zwiększenia jego wydajności. Ponadto zarówno
przełączniki jak i kontroler nie są świadome obecności dodatkowego komponentu
pomiędzy nimi, co pozwala uniknąć dodatkowego narzutu związanego z konfiguracją
sieci SDN. 

Podstawowa topologia sieci SDN wraz z umiejscowionym komponentem reprezentującym
zaimplementowane oprogramowanie (\textit{sample component}) została
przedstawiona na Rys. \ref{fig:sdn_epc_flow}. 

\begin{figure}[h]
\centering
\includegraphics[width=\textwidth]{sdn_epc_flow}
\caption{Schemat podstawowej topologii sieciowej z umiejscowionym komponentem
  implementującym rozwiązanie projektowe}
\label{fig:sdn_epc_flow}
\end{figure}

Jak zostało przedstawione na Rys. \ref{fig:sdn_epc_flow} węzły końcowe
(\textit{PC N}) komunikują się ze sobą z wykorzystaniem przełącznika sieci SDN
(\textit{SDN switch}), a cały ruch pomiędzy przełącznikiem, a kontrolerem
(\textit{SDN controller}) przemieszcza się przez wspomniany komponent
implementujący rozwiązanie projektowe (\textit{sample component}).

Topologia zaprezentowana na Rys \ref{fig:sdn_epc_flow}, jak wspomniano, jest
tolopogią najbardziej podstawową, odpowiednią do wykorzystania prezentowanego
rozwiązania. Możliwe jest również, zastosowanie tego rozwiązania w bardziej
rozbudowanej sieci, składającej się z większej liczby węzłów końcowych, 
przełączników oraz kontrolerów. W takim przypadku, można zastosować kilka
instancji równlegle działającego oprogramowania (komponentu), w celu
zwiększenia wydajności systemu, rozumianego jako kilka współpracujących ze sobą
instancji. System taki, można zastosować do większych, pracujących pod większym
obciążeniem sieci. 

\section{Wybór technik i narzędzi do implementacji systemu}