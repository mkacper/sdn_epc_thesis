\chapter{Wstęp}

Celem niniejszej pracy inżynierskiej było zaimplementowanie i wdrożenie systemu
wczesnego wykrywania ataków sieciowych typu DDoS w środowisku sieci sterowanych
programowo SDN. Najważniejszymi założeniami projektowymi były: poprawna
implementacja algorytmu do wykrywania ataku DDoS oraz możliwość skalowania
systemu celem zwiększenia jego wydajności.

Architektura systemu zaprojektowanego na potrzeby niniejszej pracy zakładała
umieszczenie stworzonego oprogramowania na południowym interfejsie sieci SDN
jako osobny jej komponent oraz, że cały ruch pomiędzy przełącznikami SDN, a
sterownikiem SDN będzie przechodził przez wspomniany komponent. Takie podejście
umożliwia umieszczenie oprogramowania na osobnej maszynie, co z kolei daje
możliwość skalowania systemu poprzez budowanie klastra złożonego z wielu
instancji komponentu, a także ułatwia konfigurację sieci (nie ma potrzeby
ingerencji w inne urządzenia sieciowe). Ponadto umiejscowienie oprogramowania
pomiędzy przełącznikami, a sterownikiem umożliwia analizowanie wiadomości
protokołu \textit{OpenFlow} wymienianych pomiędzy tymi urządzeniami. Analiza
tychże wiadomości jest wykorzystywana przez zaimplementowany algorytm wykrywania
ataku DDoS.

Algorytm służący do wykrywania ataku DDoS zaimplementowany we wspomnianym
systemie bazuje na entropii pakietów w sieci. Spadek entropii oznacza spadek
losowości pakietów. W przypadku ataku DDoS duża ilość pakietów w sieci
skierowana jest do pojedynczego węzła końcowego, powodując spadek losowości
pakietów, tym samym powodując spadek entropii. Wspomniany algorytm bazuje na
opisanej zależności. W celu obliczenia entropii analizowane są wiadomości
\textit{PACKET\_IN} protokołu \textit{OpenFlow} wysyłane przez przełączniki SDN
do sterownika SDN.

Wydajność systemu została zapewniona dzięki zastosowaniu technologii
wspierających programowanie równoległe oraz budowę systemów rozproszonych.
Wykorzystane zostały funkcyjne języki programowania Erlang oraz Elixir. Oba te
języki działają na Maszynie Wirtualnej Erlanga (\textit{BEAM}), która dostarcza
natywne wsparcie dla wspomnianych technologii. Dzięki temu stworzone
oprogramowanie może obsługiwać wiele przełączników w tym samym czasie oraz
działać w klastrze.

W celu sprawdzenia czy założenia projektowe zostały spełnione wykonano serie
testów. Sprawdzono czy proces wykrywania ataku DDoS działa prawidłowo zarówno z
wykorzystaniem jednego węzła oprogramowania, jak i rozpraszając ten proces na
kilka węzłów. Ponadto przetestowano możliwość horyzontalnego skalowania całego
systemu.