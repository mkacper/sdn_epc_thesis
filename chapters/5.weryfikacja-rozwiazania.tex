\chapter{Weryfikacja rozwiązania}

W celu weryfikacji poprawności działania zaiplementowanego systemu zostały
wykonane serie testów. System został przetestowany pod kątem sprawdzenia
pdstawowych założeń projektowych tj. poprawności wykrywania anomali, związanych
z natężeniem ruchu, w obsługiwanej sieci oraz możliwości skalowania
aplikacji \textit{sdn\_epc} w celu zwiększenia wydjaności systemu.
Przeprowadzenie testów wymagało przygotowania odpowiednich scenariuszy testowych,
zestawienia dedykowanych topolgi sieciowych oraz stworzenia rozwiązań
pozwalających na zebranie własciwych danych z testowanego systemu. Analiza
tychże wyników pozwoliła na stwierdzenie, czy i w jakim stopniu założenia
projektowe zostały spełnione. Niniejszy rozdział szczegółowo opisuje
poszczególne przypadki testowe oraz prezentuje analizę uzyskanych wyników.

\section{Test działania implementacji algorytmu}
 
Poprawność działania implementacji algorytmu służacego do wykrywania ataku
DDoS została sprawdzona w dwóch różnych przypadkach testowych, z których każdy
wykorzystywał nieco inną konfigurację testową topologi sieciowej, jak również
samej aplikacji \textit{sdn\_epc}. Przetestowane zostały przypadki, gdy:
\begin{enumerate}
  \item Ruch wygenerowany w sieci testowej był obsługiwany tylko przez jeden
    węzeł aplikacji \textit{sdn\_epc}.
  \item Ruch wygenerowanych w sieci testowej był obsługiwany przez wiele węzłów
    aplikacji \textit{sdn\_epc} działających w klastrze.
\end{enumerate}
Celem wykorzystania takich właśnie przypadków testowych było sprawdzenie
poprawności implementacji algorytmu zarówno w przypadku działania systemu jako
pojedynczy węzeł aplikacji \textit{sdn\_epc}, jak również w przypadku, gdy
system działał w klastrze. Drugi przypadek jest znacznie bardziej złożony,
ponieważ rozproszenie procesu obliczania algorytmu na wiele węzłów wprowadza
dodatkowe problemy, związane z synchronizacją stanu pomiędzy węzłami.

\subsection{Przypadek testowy z wykorzystaniem jednego węzła aplikacji
  \textit{sdn\_epc}}