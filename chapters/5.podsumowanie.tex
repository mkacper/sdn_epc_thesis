\chapter{Podsumowanie}

Na potrzeby niniejszej pracy inżynierskiej wykonano, wdrożono i przetestowano
system wczesnego wykrywania ataków DDoS w sieciach sterowanych programowo SDN.
Stworzony system spełnia wszystkie założenia projektowe tj. poprawnie wykrywa
atak DDoS w sieci SDN oraz umożliwia skalowanie celem zwiększenia wydajności.

W celu umożliwienia wykrywania ataku DDoS został zaimplementowany algorytm
bazujący na etropii (losowości) pakietów w sieci. Na potrzeby algorytmu
aplikacja \textit{sdn\_epc} analizuje wiadomości \textit{PACKET\_IN} protokołu
\textit{OpenFlow}. Wyniki analizy zapisywane są w rozproszonej Erlangowej bazie
danych \textit{Mnesia}. Dzięki temu proces obliczania algortymu może zostać
rozproszony na wiele węzłów aplikacji.

Implementacja aplikacji \textit{sdn\_epc} umożliwia działanie kilku instancji
oprogramowania jednocześnie (w klastrze). Dzięki temu możliwe jest skalowanie
aplikacji poprzez dodawanie kolejnych węzłów do klastra. Wydajnośc całego
systemu została zwiększona również dzięki odpowiedniej architekturze aplikacji
wykorzystującej programowanie równoległe.

Wdrożenie wspomianych rozwiązań było możliwe dzięki wykorzystaniu odpowiednich
technologii, które natywnie wspierają budowę systemów rozproszonych oraz
programowanie równoległe, takich jak funkcyjne języki programowania Erlang oraz
Elixir. Języki te działają na Maszynie Wirtualnej Erlanga, która dostarcza
wspomnianych funkcjonalności.

Założenia projektowe zostały sprawdzone empirycznie. Przeprowadzono serie
testów, które jedoznacznie potwierdziły, że wykonany system poprawnie wykrywa
ataki DDoS działając zarówno jako pojedynczy węzeł aplikacji \textit{sdn\_epc},
jak i w klastrze. Ponadto wykonane testy dowiodły, że wspomiany system można
skalować horyzontalnie poprzez działanie aplikacji \textit{sdn\_epc} w klastrze,
co znacząco zwiększa wydjanośc całego systemu.