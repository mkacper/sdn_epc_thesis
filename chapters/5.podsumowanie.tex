\chapter{Podsumowanie}

Na potrzeby niniejszej pracy inżynierskiej wykonano, wdrożono i przetestowano
system wczesnego wykrywania ataków DDoS w sieciach sterowanych programowo SDN.
Stworzony system spełnia wszystkie założenia projektowe tj. poprawnie wykrywa
atak DDoS w sieci SDN oraz jest możliwe jego skalowanie celem zwiększenia
wydajności.

Funkcjonalność wykrywania ataku DDoS została zaimplementowana z wykorzystaniem
algorytmu bazującego na entropii (losowości) pakietów w sieci. Na potrzeby
wspomnianego algorytmu aplikacja \textit{sdn\_epc} analizuje wiadomości
\mbox{\textit{PACKET\_IN}} protokołu \textit{OpenFlow}. Wyniki analizy
zapisywane są w rozproszonej Erlangowej bazie danych \textit{Mnesia}. Dzięki
temu proces obliczania algorytmu może zostać rozproszony na wiele węzłów
aplikacji.

Implementacja aplikacji \textit{sdn\_epc} umożliwia działanie kilku instancji
oprogramowania jednocześnie (w klastrze). Pozwala to na skalowanie aplikacji
poprzez dodawanie kolejnych węzłów do klastra. Wydajność całego systemu została
zwiększona również dzięki odpowiedniej architekturze aplikacji wykorzystującej
programowanie współbieżne i równoległe.

Wdrożenie wspomnianych rozwiązań było możliwe dzięki zastosowaniu odpowiednich
technologii, które natywnie wspierają budowę systemów rozproszonych oraz
koncepcję programowania współbieżnego i równoległego, takich jak funkcyjne
języki programowania Erlang oraz Elixir. Języki te działają na Maszynie
Wirtualnej Erlanga, która dostarcza wspomniane funkcjonalności.

Założenia projektowe zostały sprawdzone empirycznie. Przeprowadzono serie
testów, które jednoznacznie potwierdziły, że wykonany system poprawnie wykrywa
ataki DDoS działając zarówno jako pojedynczy węzeł aplikacji \textit{sdn\_epc},
jak i w klastrze. Ponadto wykonane testy dowiodły, że opracowany system można
skalować horyzontalnie poprzez działanie aplikacji \textit{sdn\_epc} w klastrze,
co znacząco zwiększa wydajność całego systemu.