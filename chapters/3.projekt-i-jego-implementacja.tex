\chapter{Projekt i jego implementacja}

Założeniami projektu opisanego w ninejszej pracy było stworzenie prototypu
oprogramowania umożliwiającego wykrywanie ataków DDoS w sieciach SDN oraz, co
ważniejsze, dającego się łatwo skalować w celu zwiększenia jego wydajności.
Wykorzystanie właściwej archtiektury rozwiązania oraz dobór odpowiednich
technologii do jego implementacji umożliwiły osiągnięcie zamierzonych celów.

\section{Architektura rozwiązania wykrywania ataku DDoS}

Projektując tytułowe rozwiązanie należało wziąć pod uwagę następujące czynniki:
\begin{itemize}
  \item umiejscowienie komponentu z oprogramowaniem w architekurze sieci SDN
  \item uzyskanie odpowiednich danych z sieci, pozwalających na wykrycie ataku
  \item wydajność działania oprogramowania
\end{itemize}

Oprogramowanie zostało zaprojektowane do działania na południowym interfejsie
architektury sieci SDN. Takie umiejscowienie dostarcza wielu możliwości.
Pierwszą z nich jest to, że oprogramowanie może być częścią przełącznika,
kontrolera lub działać na osobnym węźle. Dodatkowo oprogramowanie ma dostęp do
danych wymienianych pomiędzy przełącznikiem, a kontrolerem. Możliwość analizy
tychże danych była podstawą do implementacji algorytmu do wykrywania ataków DDoS
w sieci. Szczegóły dotycznące algorytmu zostały opisane w późniejszym rozdziale
ninejszej pracy. 

Wspomniane rozwiązanie zaimplementowano z myślą o działaniu na osobnym węźle,
ponieważ dzięki takiemu podejściu, w przypadku równoległego działania
oprogramowania, liczba jego instancji jest niezależna od liczby przełączników
lub kontrolerów. Dzięki temu, możliwe jest bardziej elastyczne skalowanie
horyzontalne oprogramowania w celu zwiększenia jego wydajności. Ponadto zarówno
przełączniki jak i kontroler nie są świadome obecności dodatkowego komponentu
pomiędzy nimi, co pozwala uniknąć dodatkowego narzutu związanego z konfiguracją
sieci SDN. 

Podstawowa topologia sieci SDN wraz z umiejscowionym komponentem reprezentującym
zaimplementowane oprogramowanie (\textit{sample component}) została
przedstawiona na Rys. \ref{fig:sdn_epc_flow}. 

\begin{figure}[h]
\centering
\includegraphics[width=\textwidth]{sdn_epc_flow}
\caption{Schemat podstawowej topologii sieciowej z umiejscowionym komponentem
  implementującym rozwiązanie projektowe}
\label{fig:sdn_epc_flow}
\end{figure}

Jak zostało przedstawione na Rys. \ref{fig:sdn_epc_flow} węzły końcowe
(\textit{PC N}) komunikują się ze sobą z wykorzystaniem przełącznika sieci SDN
(\textit{SDN switch}), a cały ruch pomiędzy przełącznikiem, a kontrolerem
(\textit{SDN controller}) przemieszcza się przez wspomniany komponent
implementujący rozwiązanie projektowe (\textit{sample component}).

Topologia zaprezentowana na Rys \ref{fig:sdn_epc_flow}, jak wspomniano, jest
tolopogią najbardziej podstawową, odpowiednią do wykorzystania prezentowanego
rozwiązania. Możliwe jest również, zastosowanie tego rozwiązania w bardziej
rozbudowanej sieci, składającej się z większej liczby węzłów końcowych, 
przełączników oraz kontrolerów. W takim przypadku, można zastosować kilka
instancji równlegle działającego oprogramowania (komponentu), w celu
zwiększenia wydajności systemu, rozumianego jako kilka współpracujących ze sobą
instancji. System taki, można zastosować do większych, pracujących pod większym
obciążeniem sieci. 

\section{Wybór odpowiednich technologii do implementacji systemu}

Podstawowym kryterium wyboru odpowiednich technologii do realizacji projektu
była wydajność. Wydajność jest rozumiana jako ilość pracy wykonana przez system
komputerowy w danym czasie i przy wykorzystaniu danej ilości zasobów
obliczeniowych \cite{distrforfunandprof}. Na potrzeby projektu, odpowiedni
poziom wydajności został zapewniony dzięki doborze techologi, które wspierają
koncepcję programowania współbieżnego, czyli możliwość wykonywania się
oprogramowania z wykorzytaniem wielu wątków fizycznych procesora maszyny, na
której zostało ono uruchomione. Ponadto, przy wyborze technologi, kierowano się
również natywnym wsparciem dla możliwosći dystrybucji oprogramowania na wiele
maszyn fizycznych, czyli, innymi słowy, wsparciem dla budowania systemów
rozproszonych. Możliwość rozbudowy systemu z wykorzystaniem wielu maszyn jest
kluczowa, ponieważ nawet dysponując technolgią ze wsparciem dla programowania
współbieżnego, istnieje granica dla rozbudowy zasobów sprzętowych, dlatego w
pewnym momencie koniecznym jest budowa systemu rozproszonego
\cite{distrforfunandprof}. 

Wykorzystanie technolgii, która umożliwa programowanie współbieżne oraz wspiera
budowę systemów rozporosznych, umożliwiło taką implementację projeku, która
pozwala na łatwe oraz efektywne jego skalowanie, celem zwiększenia wydajności.

Technologiami, które spełniają wszystkie wyżej wymienione wytyczne są funkcyjne
języki programowania Erlang oraz Elixir. Oba języki działają na Maszynie
Wirtualnej Erlanga (\textit{BEAM}). W rzeczywistości Elixir jest nowoczesną wersją
Erlanga, dopasowaną do dzisiejszych standardów. Oferuje on szereg nowoczesnych,
ułatwiających pracę programistyczną narzędzi, których Erlang nie posiada. W
rzeczywistości Elixir jest Erlangową aplikacją działającą na Maszynie Wirtualnej
Erlanga \cite{thebeambook}.

Maszyna Wirtualna Erlanga wprowadza model aktorów
\footnote{https://en.wikipedia.org/wiki/Actor\_model}, dzięki czemu posiada
natywne wsparcie dla programowania równoległego. Ponadto, \textit{BEAM} zapewnia
również wsparcie dla budowy systemów rozproszonych \cite{lyserlang}. Dzięki
temu, języki Erlang oraz Elixir są odpowiednie do budowy wydajnych,
niezawodnych, systemów rozproszonych. Z tych właśnie powodów, wspomniane
technologie zostały wykorzystane do budowy projektu przedstawionego w niniejszej
pracy.

\section{Algorytm detekcji ataku DDoS} \label{algorithm}

Alogrytm detekcji ataku, zaimplementowany na potrzeby opisywanego projektu
bazuje na entropii pakietów przepływających przez sieć SDN. Entropia pakietów
rozumiana jest jako ich losowość. Im większa entropia tym większa losowość
pakietów i odwrotnie \cite{mainddosarticle}. Entropia jest wyliczana na postawie
wzoru \ref{equ:entropy} \cite{mainddosarticle}.

\begin{equation}
H = -\sum_{i=1}^{n}p_{i}\log_{}p_{i}
\label{equ:entropy}
\end{equation}
gdzie $H$ oznacza wartość entropii, $n$ liczbę pakietów (tzw. rozmiar okna),
a $p_{i}$ wartość prawdopodbieństwa wystąpienia poszczególnego pakietu.

Wspomniany rozmiar okna jest defininowany jako ilość pakietów, dla których
została obliczona entropia. Rozważając przypadek dla okna o rozmiarze 50, przy
założeniu, że każdy element okna wystąpił dokładnie jeden raz, wartośc entropi
obliczonej zgodnie ze wzorem \ref{equ:entropy} wyniesie 1,70. Natomiast w
przypadku, gdy jeden z elementów okna pojawi się dokładnie dziesięć razy, a
pozostałe 40 elementów tylko raz, entropia wyniesie 1,47.

Przedstawiona zależność entropii i losowości pakietów została wykorzystana do
wykrywania ataku DDoS. Atak DDoS zazwyczaj skupia się na jednym, wybranym węźle
końcowym, czyli w przypadku ataku większość pakietów w sieci zawiera jeden,
konkretny adres docelowy. W takiej sytuacji losowość pakietów w sieci spada, a
co za tym idzie, spada również entropia. Bazując na tym fakcie, można ustalić
limit wartości entropii poniżej którego, uznaje się, że w sieci nastąpił atak
DDoS. 

Algorytm zaimplementowany na potrzeby opisywanego projektu analizuje adresy
docelowe pakietów \textit{IP} przepływających przez sieć, tj. zlicza ilość
wystąpień każdego adresu docelowego zawartego w badanych pakietach \textit{IP}.
Gdy ilość przeanalizwanych pakietów jest równa rozmiarowi okna obliczana jest 
entropia. W tym celu wykorzystywane są równania: \ref{equ:entropy},
\ref{equ:window} \cite{mainddosarticle} oraz \ref{equ:packet_prob}
\cite{mainddosarticle}.

\begin{equation}
W = \{(x_{1},y_{1}),(x_{2},y_{2}),(x_{3},y_{3}),...\}
\label{equ:window}
\end{equation}

\begin{equation}
p_{i} = \frac{y_{i}}{n}
\label{equ:packet_prob}
\end{equation}

Równanie \ref{equ:window} przedstawia reprezentację okna ($W$), gdzie $x_{i}$
oznacza dany docelowy adres \textit{IP}, a $y_{i}$ liczbę jego wystąpień w danym
oknie. Równanie \ref{equ:packet_prob} pozwala wyliczyć prawdopodobieństwo
danego pakietu ($p_{i}$), gdzie $y_{i}$ oznacza liczbę wystąpień danego adresu
\textit{IP}, a $n$ liczbę wszystkich pakietów w oknie ($W$).
Jeśli obliczona wartość entropii dla określonej liczby okien z rzędu
(rekomendowana liczba okien  wynosi 5 \cite{mainddosarticle}) jest mniejsza niż
zadana wartość, uznaje się, że w sieci nastąpił atak. 

\section{Aplikacja wykrywania ataku DDoS}

Aplikacja \textit{sdn\_epc} \footnote{https://github.com/mkacper/sdn\_epc}
(\textit{Software Defined Networking Elixir Pseudo Controller}) została w
głównej mierze wykonana przy użyciu języka programowania Elixir, aczkolwiek
wykorzstuje również kod napisany w Erlangu (głównie do interakcji z zewnętrznymi
Erlangowymi bibliotekami). Użycie dwóch języków jednocześnie było możliwe,
ponieważ, jak wspomniano we wcześniejszym rozdziale, oba te języki finalnie są
kompilowne do tego samego kodu binarnego i wykonują się na tej samej maszynie
wirtualnej (\textit{BEAM}). Ponadto, kompilator Elixira potrafi również
kompilować kod Erlanga.

Jak przedstawiono na Rys. \ref{fig:sdn_epc_flow} strona
\pageref{fig:sdn_epc_flow} aplikacja \textit{sdn\_epc} została zaprogramowana
jako osobny komponent, działający na osobnej maszynie, połączony bezpośrednio z
przełącznikami oraz z kontrolerm sieci SDN. Komponenty te komunikują się ze
sobą z wykorzystaniem protokołu \textit{OpenFlow}. W przedstawionej konfiguracji,
cały ruch sieciowy pomiędzy przełącznikami, a kontorelerem przepływa przez
komponent \textit{sdn\_epc}. Umożliwienie połączenia systemu Elixirowego z
komponentami sieciowym, z wykorzystaniem \textit{OpenFlow} wymaga umiejętności
obsługi tego protokołu przez aplikację \textit{sdn\_epc}. W tym celu
wykorzystane zostały Erlangowe biblioteki \textit{of\_protocol}
\footnote{https://github.com/FlowForwarding/of\_protocol} oraz
\textit{ofs\_handler} \footnote{https://github.com/FlowForwarding/ofs\_handler}.

Przepływający przez komponent \textit{sdn\_epc} ruch \textit{OpenFlow} jest
analizowany na potrzeby algorytmu opisanego w rozdziale \ref{algorithm}.
Analizowane są wiadomości \textit{PACKET\_IN} protokołu \textit{OpenFlow}, a
konkretnie docelowe adresy pakietów \textit{IP} przesyłanych jako część
wiadomości \textit{PACKET\_IN}. Wyniki zapisywane są w rozproszonej bazie danych
(\textit{Mnesia}) \footnote{hhttp://erlang.org/doc/apps/mnesia/index.htm}
dostarczanej razem ze standardową dystrybucją Erlanga. Następnie, zapisane
wyniki wykorzystywane są do działania wspomnianego algorytmu.


% jak działą, switch <-> sdn_Epc <-> ctrl, openflow
% ze analizuje PACKET_IN
% że jest switch per procs
% że da się klastrować
% że problem z synchronizacją stanu, mnesia