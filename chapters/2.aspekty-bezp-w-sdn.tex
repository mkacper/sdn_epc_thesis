\chapter{Wybrane aspekty bezpieczeństwa w sieciach SDN }

Ostatnimi czasy, sieci sterowane programowo (SDN) zdobywają popularność zarówno
w środowiskach akademickich, jak również wśród komercyjnych przedsiębiorstw.
Zcentralizowany model zarządzania siecią, jaki wprowadzono w architekturze SDN
fundamentalnie zmienił spojrzenie na sposób zarządznia w sieciach rozproszonych
\cite{ddosNYarticle}. Sieci SDN dostarczają nowych możliwości w zakresie
monitorowania sieci, a co za tym idzie, również w zakresie wykrywania ataków
sieciowych, m.in. ataków typu DDoS (z ang. \textit{distributed
  denial-of-service}). Niniejszy rozdział przybliża ideę sieci SDN oraz wybrane
sposoby wykrywania ataków DDoS. 

\section{Sieci sterowane programowo SDN}
Sieci sterowane programowo (z ang. \textit{Software Defined Networking})
wprowadzają rozdział warstwy danych od warstwy zarządzającej oraz wprowadzają
centralny punkt zarządzania siecią, który jest w pełni programowalny \cite{onf}.
Innymi słowy, możliwe jest zaprogramowanie konfiguracji sieci. Warstwa danych,
jest odpowiedzialna tylko i wyłącznie za przełączanie danych, wedle reguł
otrzymanych od warstwy zarządzającej. Do warstwy danych należą przełączniki SDN.
Warstwa zarządzająca natomiast, jest odpowiedzialna za podejmowanie wszelkiego
rodzaju decyzji związanych z działaniem sieci, np. określa jak powinny być
przełączane/rutowane pakiety w sieci. Urządzeniem warstwy zarządzającej jest
kontroler SDN. Kontroler komunikuje się z przełącznikami z wykorzystaniem tzw.
południowego interfejsu \cite{sdninterfaces}.

Komunikacja pomiędzy przełącznikami, a kontrolerem w sieci SDN odbywa się z
wykorzystaniem protokołu \textit{OpenFlow}. Jest on obecnie szeroko stosowany w
dzisiejszych sieciach SDN \cite{ddoskoreaarticle}. Przełącznik SDN przełącza
pakiety zgodnie z tablicą przepływów (z ang. \textit{flow table}). Przepływ
charakteryzuje pewną grupę podobnych pakietów, np. mających taki sam adres
docelowy \textit{IP}. Każda reguła w tablicy przepływów określa jaką akcję
przełącznik powinien podjąc w związku z danymi należącymi do danego
przepływu, np. przełączyć je na port X. Tablica przepływów jest zarządzna przez
kontroler sieci SDN.

Przełącznik oraz kontroler komunikują się ze sobą z wykorzstaniem
predefiniowanych wiadomości protokołu \textit{OpenFlow}. W przypadku, gdy
przełącznik nie znajduje reguły w tablicy przepływów pasującej do ramki/pakietu,
którą otrzymał, wysyła wiadomość \textit{PACKET\_IN} protokołu
\textit{OpenFlow}, do kontrolera. Kontroler podejmuje decyzję co zrobić z danym 
pakietem/ramką i przekazuje tę informację do przełącznika za pomocą wiadomości
\textit{PACKET\_OUT}. Następnie instaluje nowy przepływ w tablicy przepływów
przełącznika, aby mógł on przełączać podobne pakiety/ramki bez konieczności
udziału kotrolera. W tym celu kontroler wysła do przełącznika wiadomość
\textit{FLOW\_ADD}. Protokół \textit{OpenFlow} definiuje o wiele więcej
wiadomości, jednakże nie ma konieczności ich omawiania na potrzeby niniejszej
pracy inżynierskiej.

\section{Bezpieczeństwo sieci SDN w kontekście ataku DDoS}

Sieci SDN są nadal stosunkowo nową koncepcją, która nie jest jeszcze w
powszechnym użyciu, ale wzraz ze wzrostem zaintersowania tą technologią oraz
powszechności użycia, sieci SDN staną się w niedelakiej przyszłości celem ataków
\cite{sdnsecurityblog}. W tym podrozdziale zostanie omówiona kwestia
bezpieczeństwa sieci SDN w kontekście ataku DDoS.

Atak DDoS jest rodzajem ataku sieciowego, w którym atakujący wykorzystuje wiele
węzłów sieciowych np. zainfekowanych komputerów do wygenerowania ruchu
sieciowego adresowanego do konkretnego węzła końcowego. W efekcie atakowany
węzeł wykazuje większe latencje lub w ogóle przestaje odpowiadać na żądania,
ponieważ jest całkowicie zaabsorbowany obsługą fałszywego ruchu.

W przypadku sieci SDN, atak DDoS powoduje dodatkowe problemy w sieci. W
niektórych pracach naukowych atak ten jest nazywany \textit{SDN-DDoS}
\cite{ddosbronksarticle}: sieć SDN jest zalana ruchem, który nie należy do
żadnego znanego przełącznikom w sieci przepływu (jest wręcz losowy). W takiej
sytuacji, każdy przełącznik, który obsługuje fałszywy ruch wysyła wiadomości
\textit{PACKET\_IN} do kontrolera w celu obsługi nieznanych (fałszywych)
pakietów. Taka sytuacja powoduje szereg implikacji: prawdziwy ruch sieciowy jest
obsługiwany wolniej lub w ogóle, ponieważ kontroler jest zajęty przetwarzaniem
fałszywego ruchu \cite{indiaarticle}. Ponadto, do kontrolera jest wysyłana duża
liczba wiadomości \textit{PACKET\_IN}, co może doprowadzić do jego przeciążenia.
W przypadku awarii kontrolera cała sieć SDN przestaje być użyteczna
\cite{ddoskoreaarticle}. Z wyż. wym. powodów wczesne wykrycie ataku DDoS w
sieciach SDN ma kluczowe znaczenie dla poprawnego działania całej sieci. 

\section{Metody wykrywania ataków DDoS w sieciach SDN}

Ataki DDoS w kontekście sieci SDN budzą duże zaintersowanie w środowisku
akademickim. Pojawiło się dość sporo artykułów naukowych prezentujących rozmiate
metody wykrywania tego typu ataków. Opisane zostały zarówno bardzo podstawowe
metody bazujące na entropii, jak również te bardziej zaawansowane. Na potrzeby
niniejszej pracy przywołano i pokrótce opisano wybrane metody wykrywania
ataków DDoS. 

Jedna z metod przedstawiona w \cite{ddosNYarticle} opisuje sposób wykrywania
ataku bazujący na monitorowaniu natężnia ruchu dla poszczególnych przepływów, a
także ich asymetrii, tzn. monitorowania natężnia ruchu w obie strony: zarówno
ruchu do potencjalnej ofiary, jak i od niej. Dzięki takiemu podejściu możliwe
jest odróżnienie naturalnych przepływów o wysokim natężneiu np. transfer danych
pomiędzy centrami danych od przepływów odpowiedzialnych za atak DDoS.
Wykorzystanie wspomnianej metody opisano na dwa sposoby, jako \textit{Metodę
  Sekwencyjną} oraz \textit{Metodę Równoległą}. 

Kolejna metoda zaprezentowana w \cite{ddoskoreaarticle} bazuje na czynnikach
związanych z czasem. Badacze we wspomnianym artykule zaprojektowali metodę
wykrywania ataku DDoS wykorzystującą ilość czasu jaki upłynął zanim ruch
osiągnął pewien stopień natężenia, wzorce czasowe ataków DDoS oraz docelowe
adresy pakietów w sieci. Wspomniana ilość czasu związana z natężeniem ruchu jest
wykorzystywana do wykrycia ataku DDoS, natomiast wzorce czasowe mają zapobiegać
atakom w przyszłości. Metody wykorzystujące informację o długości trwania ataku
są rzadko stosowane \cite{ddoskoreaarticle}. 

Metoda wykrywania ataków DDoS bazująca na entropii została opisana w
\cite{mainddosarticle}. Entropia jest obliczana na podstawie docelowych adresów
\textit{IP} poszczególnych pakietów w sieci. Jest ona obliczana dla zadanej
długości okna, które składa się z ustalonej liczby pakietów. Gdy obliczna
entropia spadnie poniżej zadanego poziomu dla kilku następujących po sobie okien
uznaje się, że w sieci nastąpił atak. Algorytm ten został zaimplementowany w
projekcie wykonanym na potrzeby niniejszej pracy inżynierskiej. Został on
szczegółowo opisany w rozdziale \ref{algorithm} strona \pageref{algorithm}.

System wykrywania ataków DDoS opisany w \cite{bloomarticle} jest w stanie wykryć
typ ataków skierowanych na poszczególne łącze w sieci. Tego typu atak ma na celu
wysycenie konkretnego łącza. Proponowane rozwiązanie bazuje na analizie tablic
przepływów oraz pakietów w sieci SDN. Opisywany system składa się z dwóch
elementów: \textit{Collector}'a oraz \textit{Detector}'a. \textit{Collector} ma
za zadanie skanować tablice przepływów w sieci w celu znalezienia podejrzanych
przepływów (odpowiedzialnych za wysycenia łącza). Podejrzane przepływy są
zapisywane w specjalnej strukturze danych, na potrzeby której wykorzystano Filtr
Blooma \footnote{https://en.wikipedia.org/wiki/Bloom\_filter}.
Odpowiedzialnością \textit{Detecor}'a jest skanowanie sieci w celu pozyskania
pakietów do analizy. Następnie komponent sprawdza, czy dany pakiet należy do
któregoś z podejrzanych przepłyów i wysyła odpowiednie powiadomienie do
kontrolera SDN.